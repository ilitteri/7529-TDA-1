\documentclass[../main.tex]{subfiles}

\begin{enumerate}
    \item Hasta el momento hemos visto 3 formas distintas de resolver problemas. Greedy, división y conquista, y programación dinámica..
    \begin{enumerate}
        \item Describa brevemente en qué consiste cada una de ellas.
        \item Identifique similitudes, diferencias, ventajas y desventajas entre las mismas. ¿Podría elegir una técnica sobre las otras?
    \end{enumerate}
    \item Tenemos un problema que puede ser resuelto por un algoritmo Greedy (G) y por un algoritmo de Programación Dinámica (PD). G consiste en realizar múltiples iteraciones sobre un mismo arreglo, mientras que PD utiliza la información del arreglo en diferentes subproblemas a la vez que requiere almacenar dicha información calculada en cada uno de ellos, reduciendo así su complejidad; de tal forma logra que O(PD) < O(G). Sabemos que tenemos limitaciones en nuestros recursos computacionales (CPU y principalmente memoria). ¿Qué algoritmo elegiría para resolver el problema?
\end{enumerate}

\textit{Pista: probablemente no haya una respuesta correcta para este problema, solo justificaciones correctas.}