\documentclass[../main.tex]{subfiles}

Dada la siguiente relación de recurrencia

\begin{equation*}
    a \cdot T\left(\frac{n}{b}\right) + \mathcal{O}(c)
\end{equation*}

Con:

\begin{equation*}
    \begin{aligned}
        a&: 1 + \text{(los dos dígitos del padrón de la izquierda mod 9)}&\\
        b&: 2 + \text{(los dos dígitos del padrón de la izquierda mod 7)}&\\
        c&: \text{“n” si su padrón es múltiplo de 4,}&\\
         &   \text{sino “$n\log(n)$” si su padrón es múltiplo de 3,}&\\
         &   \text{sino “n2”}&
    \end{aligned}
\end{equation*}

Se pide:

\begin{enumerate}
    \item Responda y complete: ¿Qué le falta a la relación de recurrencia para que se pueda aplicar el teorema maestro?
    \item Calcular la complejidad temporal utilizando el teorema maestro.
    \item Explique paso a paso cómo llega a la misma.
\end{enumerate}