\documentclass[../main.tex]{subfiles}

Dada los siguientes números (completada por su número de padrón)

\begin{equation*}
    \begin{aligned}
        a35b411c\\
        2d98ef55
    \end{aligned}
\end{equation*}

con:
\begin{align*}
    &\text{a: dígito del padrón correspondiente a la unidad}\\
    &\text{b: dígito del padrón correspondiente a la centena}\\
    &\text{c: los dos dígitos del padrón de la izquierda mod 7}\\
    &\text{d: dígito del padrón correspondiente a la decena}\\
    &\text{e: dígito del padrón correspondiente a la unidad de mil}\\
    &\text{f: los dos dígitos del padrón de la derecha mod 9}\\
\end{align*}

Ejemplo. Padrón: 95473

\begin{equation*}
    \begin{aligned}
        33544114\\
        27985155
    \end{aligned}
\end{equation*}

Se pide:

\begin{enumerate}
    \item Resuelva la multiplicación paso a paso utilizando el algoritmo de Karatsuba.
    \item Cuente la cantidad de sumas y multiplicaciones que realiza y relaciónelo con la complejidad temporal del método.
    \item Comparar lo obtenido con el método de multiplicación tradicional. ¿Observa alguna mejora? Analice.
    \item ¿Por qué se puede considerar al algorítmo de Karatsuba como de "división y conquista"?
\end{enumerate}