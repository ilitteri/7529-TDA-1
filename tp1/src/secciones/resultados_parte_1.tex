\documentclass[../main.tex]{subfiles}

\subsection{Preliminar}

Dada los siguientes números (completada por su número de padrón)

\begin{equation*}
    \begin{aligned}
        a35b411c\\
        2d98ef55
    \end{aligned}
\end{equation*}

con:
\begin{align*}
    &\text{a: dígito del padrón correspondiente a la unidad}\\
    &\text{b: dígito del padrón correspondiente a la centena}\\
    &\text{c: los dos dígitos del padrón de la izquierda mod 7}\\
    &\text{d: dígito del padrón correspondiente a la decena}\\
    &\text{e: dígito del padrón correspondiente a la unidad de mil}\\
    &\text{f: los dos dígitos del padrón de la derecha mod 9}\\
\end{align*}

Con mi padrón: 106223, las letras serían

\begin{align*}
    a &= 3\\
    b &= 2\\
    c &= 10 \bmod 7 = 3\\
    d &= 2\\
    e &= 6\\
    f &= 23 \bmod 9 = 5
\end{align*}

Luego reemplazando los valores de las letras en los números:\

\begin{equation*}
    \begin{aligned}
        33524113\\
        22986555
    \end{aligned}
\end{equation*}

% \subsubsection*{Algoritmo de Karatsuba}

% Éste algoritmo es un algoritmo de división y conquista. En resúmen, se parte en dos los números a multiplicar recursivamente, realiza las operaciones requeridas, luego une los resultados para obtener el resultado de la multiplicación. 

\subsection{Multiplicación usando el algoritmo}
\subfile{../subsecciones/paso_a_paso.tex}

\subsection{Árbol de recursión}
\subfile{../subsecciones/arbol_de_recursion.tex}

\subsection{Cantidad de sumas y multiplicaciones computadas con el algoritmo}
\subfile{../subsecciones/cantidad_sumas_multiplicaciones_algoritmo.tex}

\subsection{Multiplicación usando el método tradicional}
\subfile{../subsecciones/metodo_tradicional.tex}

\subsection{Cantidad de sumas y multiplicaciones computadas con el método tradicional}
\subfile{../subsecciones/cantidad_sumas_multiplicaciones_tradicional.tex}