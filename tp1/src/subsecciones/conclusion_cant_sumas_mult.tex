\documentclass[../main.tex]{subfiles}

Analizando la cantidad de sumas y multiplicaciones computadas en el algoritmo, se puede concluir que, en este caso, el multiplicando y el multiplicador no son lo suficientemente grandes como para que sea conveniente emplear éste algoritmo para realizar su multiplicación.
A priori se puede decir que éste algoritmo es óptimo y eficiente (porque aún no se comparó con otro, más adelante en este trabajo lo compararemos con el método tradicional y concluiremos si sigue siendo eficiente o no).

\subsubsection{Teorema maestro}

\begin{equation*}
    \boxed{T(n) = aT\left(\frac{n}{b}\right) + \mathcal{O}(n^c)}
\end{equation*}
\begin{equation*}
    \begin{aligned}
        &< & &\rightarrow T(n) = \mathcal{O}(n^{c})\\
        Si: \log_{b}(a) \ &= &C &\rightarrow \mathcal{O}(n^{c} \log_{b}(n)) = \mathcal{O}((n^{c}\log(n))\\ 
        &> & &\rightarrow \mathcal{O}(n^{\log_{b}(a)})\\
    \end{aligned}
\end{equation*}

\subsubsection{Análisis de complejidad del algoritmo de Karatsuba con el teorema maestro}

Ya que las sumas, las restas, y los desplazamientos de dígitos (multiplicaciones por potencias de la base) en el caso base del algoritmo de Karatsuba requieren tiempos proporcionales a $n$, su coste se hace insignificante a medida que crece $n$. Precisamente, si $T(n)$ denota el número total de operaciones elementales que el algoritmo realiza cuando se multiplican dos números de $n$ dígitos, entonces podemos escribir:

\begin{equation*}
    \begin{aligned}
        a &= 3 \quad \because T\left(\frac{n}{2}\right) \text{ es llamada 3 veces en el algoritmo}\\
        b &= 2 \quad \because \text{ se divide en 2 el arreglo inicial}\\
        c &= 1 \quad \because \text{ calcular el resultado de la expresión es} \mathcal{O}(n)
    \end{aligned}
\end{equation*}

Reemplazando en el Teorema Maestro:

\begin{equation*}
    T(n) = 3T\left(\frac{n}{2}\right) + \mathcal{O}(n)
\end{equation*}

luego $\log_{b}(a) = 1.584962501 > c = 1$ entonces obtenemos la cota superior asintótica:

\begin{equation*}
    \boxed{T(n) = \mathcal{O}(n^{\log_{b}(a)})  \approxeq \mathcal{O}(n^{1.58})}
\end{equation*}

Esto quiere decir que éste algoritmo es una mejor opción que el del método tradicional (teóricamente, pues $n^{1.58} < n^{2}$).